\documentclass[12pt]{article}

\usepackage[absolute,overlay]{textpos}
\setlength{\TPHorizModule}{1mm}
\setlength{\TPVertModule}{1mm}

\usepackage{subfigure}
\usepackage[T1]{fontenc}
\usepackage[latin9]{inputenc}
\usepackage{geometry}
\geometry{verbose,tmargin=2cm,bmargin=2cm,lmargin=2cm,rmargin=2cm}
\usepackage{textcomp}
\usepackage{amsmath}
\usepackage[authoryear]{natbib}
\usepackage{multicol}
\usepackage{epsfig}
\usepackage{graphicx}
\usepackage{amssymb,amsmath}
\usepackage{url}
\usepackage{multimedia}
\usepackage{hyperref}
\usepackage{lmodern}
\usepackage{times}
\usepackage{tikz}
\usepackage{amsmath}
\usepackage{amsthm}
\usepackage{verbatim}
\usepackage{multirow}
\usepackage{blindtext}
%%%%%%%%%%%%%%%%%%%%%%%%%%%%%% User specified LaTeX commands.
\usepackage{babel}
\usepackage{color}
\newcommand{\carlos}[1]{\textcolor{red}{#1}}
\usepackage{babel}

%latex whitepaper style?
%MaryVilla VeloLab MindLab DeepLab LabLand Cryptovelo\\
\title{\large{Decentralized ebike-energy-to-grid platform \\$\mathcal{DEEP}$
    %From autonomy to power grid
  }}
\author{}
    \maketitle
%\begin{center}
  %\hspace{-0.1 in}
%  \includegraphics[weight=18cm,height=18cm]{VelomindsLogo.png}
%\end{center}
    \begin{document}

\newpage
\tableofcontents
\newpage

    \section*{Abstract}
    Green power generation, storage and distribution will be composed
    of millions of small, decentralised power sources of producers and
    consumers, the prosumers. In such systems, it will be important to
    connect efficient green power devices to batteries to secure and
    instantaneous, autonomous green energy transactions across
    prosumers as energy market conditions change. In this proposal we
    aim to 1) quantify green power production of pedal turbine
    generators for bike networks of different sizes, and 2) integrate
    green power production to a decentralized ebike energy-to-grid
    platform (DEEP) using blockchain technology. We propose an
    integrated smart-testnet mini-grid prototype connecting pedal
    turbine generators to open-source energy blockchain platforms
    containing producers and consumers to study the feasibility and
    scalability of ebike-to-energy networks. We outline the
    bottlenecks, the improvements needed, and a roadmap for the future
    of coupling efficient pedal power turbine generators to decentralized
    open-source computer and power networks.
    \\
    Keywords: Green energy. Interconnected networks. Plug-in electric
    bikes. ebike networks. Smart mini-grid. Computer networks. Power
    network. Blockchain. Energy harvesting. Micro-energy storage.

    \newpage

    \section*{Lay summary}

    ------------------------ \\
    Integrating pedal power turbine generator-battery to blockchain
    energy platforms (i.e., Hyperledger fabric, Tobalaba, Grid+)
    \\
    \\
    ---Transdisciplinarity------------------------------------------------------------------------------------------- \\
    Data science and modeling: Efficiency curves integrating turbine generator speed (rpm) vs. Output power (W) vs. riding time\\
    (proxy of generator speed) vs. Power (W)  bicycle dynamic models, thermodynamics constraints\\
    Engineering and design: oriented harvesting energy\\
    Computer science: Distributed and blockchain open-source software, for example the VESC open-software \\
    Electric engineer: Designing pedal power turbine generators to connect them to dual battery, inverters and micro-grids\\
    ------------------------------------------------------------------------------------------------------------------
    \\
    \\
    PROS
    \\
    0. Reduce $C0_{2}$ across the energy cycle\\
    1. Green energy --- sustainable development \\
    2. Local smart grid development \\
    3. Bottom-up: distributed individual-community prosumers \\
    4. Urban bike network development and functional e-mobility networks \\
    5. Research about efficiency and smart metrics by tracking and sharing energy production-consumption data \\
    6. Real time price in the green energy markey by connecting many-to-many battery types-decentralized-green energy platforms \\
    7. Deregulation energy production many countries \\
    8. Frame architecture-design oriented to produce energy \\
    \\
    CONS
    \\
    0. Gear box turbines not developed for the bicycle industry\\
    1. Low efficiency dual battery\\
    2. Low energy recovery\\
    3. Energy production mostly for large kms/day \\
    4. Absence of infra like inverter-energy-to-grid networks in urban landscapes

%\end{document}
    
\newpage
   
\section{Industry overview}

% Check\\  
%Hyperledger Fabric

%The problem--High dependency of a few and highly centralized energy
% sources like fossil...
Cities worldwide are under a fast transformation of
mobility. Decreasing emissions and congestion are two of the main
engines of this rapid transformation and making mobility networks
functional while increasing efficiency of e-mobility is one of the
first challenges to make it happen. Innovation of e-mobility vehicles
is occurring at an unprecedented rates. Many new technologies are
being integrated to increase efficiency, autonomy, and connectivity of
e-mobility vehicles. This synthesis between rapidly transforming
mobility networks and first generation e-mobility vehicles is a first
step towards the functionality of mobility networks. The second
generation of e-mobility vehicles is yet to be envisioned.

Deregulation of energy production is occurring at many countries and
considering mobility networks not only as reducing C02 emisions and
congestion but as a decentralized energy production networks is an
open issue. Despite a great potential to integrate engines, generators
and batteries of e-mobility vehicles to smart-mini-grid using
open-source energy platforms, energy production oriented e-mobility
devices are at a very incipient stage. Here we introduce two designs
for e-mobility, one focused on autonomy and the second one in
autonomy, energy production and storage. We provide estimations of
efficiency, autonomy and power capacity to discuss the pros and cons
of coupling energy-production driven e-mobility vehicles, mobility
networks and energy-to-grid networks.

Hundreds of millions of people have attained modern energy access over
the last two decades through centralized distribution networks
(refs). This means that more people on Earth than ever before are now
connected to ever-growing and interconnected power networks. How we
power these networks in the next decade will determine much of the
impact of humanity on Earth named global warming and biodivesity loss
(refs).

First generation e-mobility devices and rapidly expanding mobility
networks to reduce emisions and congestion in many cities...

Second generation e-mobility devices coupling functional mobility
networks to energy grids to reduce centralization and open energy
markets to prosumers... regulation towards decentralized markets? Is
it viable to produce energy with electric vehicles like bicycles?
Which are the existing numbers and the main scalability issues? How
will the new technologies and their integration change the existing
numbers?

Green energy storaged devices powering these grids (Solar and wind
power, electric vehicles to grid -- EV2G), will be an important
component to diminish human impact on Earth. Millions of small,
decentralised power producers and consumers will have ample choice
options. Electric vehicles, solar panels, wind, ebikes...etc will be
endpoints interacting with each other in form of microgrids connected
to sensors and energy management platforms. In such systems, it will
be important to secure green energy production, storage and
distribution by verifying instantaneous, decentralized and autonomous
transactions across these nodes as green energy demand changes.

However, the transition from the first generation e-mobility devices
driven by increasing autonomy and connectance coupled with functional
mobility networks to the second generation, e-mobility vehicles to
grid is at a very incipient stage. Plug-in electric bikes can
potentially be green storage devices. Yet, many questions remain about
its power, efficiency and scalability, and how the integration among
new technologies (i.e., energy efficiency like backward pedaling...,
integrated solar panels) and prototypes designed to produce and store
energy will change existing power and efficiency and its relationship
to explore autonomy and increase scalability.

Plug-in electric bikes (PEB) as green storage devices will be part of
these mobility nerworks. There are, at least three componentsthat need
to be integrated to make PEB feasible: 1) ebikes are currently built
and marketed as a personal transportation solution, not as a personal
plug-in power generator or a energy storage solution. Hubs, motors and
frames are not designed with EV2G functionality in mind; 2) Physical
infrastructure in the form of a network of dual-energy stations in
places such as bicycle parkings and workplaces will be required, and
3) Managing decentralized energy generation and storage accounting for
information collection and control of charging/discharging of PEB will
be key to distribute green energy from PEB. For example, a platform
accounting for individual prosumer behaviour and market signals from
the national or international electricity market (refs) (Fig 1).

Here we introduce a blockchain platform connecting battery sensors
from a PEB to an energy management software. The blockchain is what
makes simultaneous exchange between producers and consumers possible
in the first place, and is thus the required link to a certified
CO2-free energy world (Box 1). We introduce a case study of a
decentralized green energy to blockchain platform currently in testnet
phase. Finally we summarize projects building transparent platforms
for clean flow of renewable energy from the grid to homes and from
homes to the grid (refs).

Ultimately, as dynamic distributed energy markets become mainstream,
the owners of decentralized devices can earn an income, not just from
the energy they sell but from the network services they provide such
as frequency and voltage control, load shifting, load shaping and load
sinking.

\section{Mobility networks}

\subsection{First generation e-mobility: autonomy and connectivity}

Integration of technologies to increase autonomy and connectivity and
reduce emissions and congestion

\subsection{Second generation e-mobility: technology integration and scalability}

Integration of technologies to reduce energy centralization and
increase scalability

\subsection{Energy-to-grid platforms}

EV2G technology has been in the market since the last decade.

E2GV networks are already functional -- give examples. EV examples and
existing companies providing dual energy algorithms ...Connecting to
the grid -- pros/cons costs--distribution in the grid--performance
devices--efficiency loss


https://utilitymagazine.com.au/ev2g-imminent-reality-or-electric-fiction/

The key to grid integration of EV�based storage is the aggregation of
individual vehicles into a critical mass of total storage, with a
combined capacity to palpably impact the grid. Individual storage
capacities of single cars for grid integration are currently in the
range of only 1 to 95kWh or 0.2 to 19kW output over a five hour
discharge of full capacity

However, there are many informational uncertainties and infrastructure
voids which would need to be overcome to make EV2G feasible. Physical
infrastructure in the form of a network of chargers in places such as
car parks and workplaces will be required. In addition to physical
infrastructure, EV2G systems will require significant informational
infrastructure to operate. This will require the construction of a
deep information layer, including data on individual EVs, their
parking status, storage capability, power flows, state of charge, and
the preferences and constraints of car owners and these are just some
of the datasets required to effectively realise EV2G

Decentralizing energy production from individuals to the blockchain
technology -- examples of companies

\subsection{Regulation}

Regulations that might affect mobility networks

\subsubsection{Deregulating energy markets}
Deregulated energy markets

\subsubsection{Regulating EV limits but energy production?}
regulation EV -- speed limits but energy production?


Plug-in electric bikes (PEB) belongs to a broader class of EV2G but
with much more speed limit (max energy per capita USA law is < 20 mph
unassisted and < 750 watt motor = bicycle). The most influential
definition which distinguishes which e-bikes are pedelecs and which
are not, comes from the EU directive (EN15194 standard) for motor
vehicles, a bicycle is considered a pedelec if the pedal-assist,
i.e. the motorised assistance that only engages when the rider is
pedalling, cuts out once 25 km/h is reached, and when the motor
produces maximum continuous rated power of not more than 250 watts
(n.b. the motor can produce more power for short periods, such as when
the rider is struggling to get up a steep hill).

An e-bike conforming to these conditions is considered to be a pedelec
in the EU and is legally classed as a bicycle. The EN15194 standard is
valid across the whole of the EU and has also been adopted by some
non-EU European nations and also some jurisdictions outside of Europe
(such as the state of Victoria in Australia).

Pedelecs are much like conventional bicycles in use and function the
electric motor only provides assistance, most notably when the rider
would otherwise struggle against a headwind or be going
uphill. Pedelecs are therefore especially useful for people living in
hilly areas where riding a bike would prove too strenuous for many to
consider taking up cycling as a daily means of transport. They are
also useful when it would be helpful for the riders who more generally
need some assistance, e.g. for elderly people

In summary pedelecs have pedal-assist only, motor assists only up to a
decent but not excessive speed (usually 25 km/h), motor power up to
250 watts.

\section{EV: from autonomy to production and storage design}


\subsection{Estimations autonomy}

Data available about the time evolution of autonomy -- is it increasing?

\subsection{Estimations per capita energy production}

Estimations existing er capita energy production 
Estimations future energy production integrating efficiency and new engines

Max watts per hour for a 250 watts motor 
Battery performance and efficiency: battery types

(Li-ion or Li-air batteries and plug-in devices contain pros/cons that
need to be taken into account to accurarely estimate efficiency and
per-capita energy production. See Fig 1: historical efficiency of
batteries and plug in devices);

scalability

\subsection{EV designed for maximum autonomy}

       Carry on the plan\\
       Quantitative assessment\\
       Kw consumed to Kw produced (for example Tesla vs velo)\\
       Ratio Kw_$cons$/Kw_$pro$ vs rentability?\\
       Evolution efficiency-cycles-size batt\\
       Km/h vs. Kw vs. efficiency\\
       Kw vs. efficiency $$(market)\\  
       Solution\\ 
       Go backwards\\
       Is it correct? Reproducible?\\
       Apply to other problems? Data?

\subsection{EV designed for energy production and storage}

       Carry on the plan\\
       Quantitative assessment\\
       Kw consumed to produced Kw (for example Tesla vs velo)\\
       Ratio Kw_$cons$/Kw_$pro$ vs rentability?\\
       Evolution efficiency-cycles-size batt\\
       Km/h vs. Kw vs. efficiency\\
       Kw vs. efficiency vs. $$(market)\\  
       Solution\\ 
       Go backwards\\
       Is it correct? Reproducible?\\
       Apply to other problems? Data?

\section{Testnet: Decentralized ebike-energy-to-grid platform (DEEP)}

Universal-battery app
Hub-battery specific app

Decentralized energy markets connect homes and electric vehicles with
software that automatically sells and buys power to and from the grid
on the basis of real-time price signals. Tracking of renewable-energy
certificates is one of dozens of potential applications of blockchain
technology that could solve data management challenges in the
electricity sector. Blockchain allows secure verifiability and
transparency of the energy transactions using small-scale
batteries. This ensures a simple connection between suppliers of
locally distributed flexible energy producers and consumers demanding
power locally or regionally from the grid.

Many platforms currently allow for setting up a blockchain framework
to track green energy in the power grid. We will implement two
testnets, one using the Hyperledger Fabric, one of the Hyperledger
projects hosted by The Linux Foundation, and a second one using
Tobalaba, a testnet hosted by the Energy Web Foundation
(http://energyweb.org/network/). Here we introduce a testnet to
plug-in a small number of ebikes to the grid accounting for an
ID-renewable-energy certificate, spatial location of the transaction
() and metrics to characterize the stability of the power grid. The
steps are the following:

1. Connect hub motor and batteries to an 
open-software and interface

Examples http://vedder.se/2015/01/vesc-open-source-esc/
https://www.kickstarter.com/projects/800872710/os-ebike-open-source-electric-bicycle-design-you-c/description

A VESC is at the heart of the bike and handles the BLDC (brushless
direct current) motor control, receives input from the throttle or
pedal sensors, and outputs data to the display to show speed,
distance, battery statistics, or anything else you'd like to view
while riding. 

 
2. Plugin hub battery using VESC to the grid (plugin energy converted
required). Check grid features.  3. Login deep app to connect to the
bluetooth ready battery. Go to renewable-energy certificate (REC) to
generate the ID and kw of energy to upload 4. Connect to P2P trading,
tobalaba or Hyperledger fabric 5. You can choose between A. buyers and
kw price or B. upload energy to the grid using the ID 6. Link the ID
to your account and transfers will be made in "real" time



App connecting hub battery to tobalaba grid+ or ibm - open-source energy blockchain platform

Own crypto
https://www.ethereum.org/token

Grid+
https://blog.gridplus.io/gridx-the-future-of-energy-markets-da104c285363?gi=3e04edc41244

Grid singularity
http://gridsingularity.com/#/0/1

Tobalaba
http://energyweb.org/network/


Power ledger

LO3

arcc 
batripower
neomou

\section{Roadmap}

https://blockhive.ee/ilp

\begin{itemize}
\item
\item 
\end{itemize}

\section{Boxes} 


\caption{Box 1: What can add the blockchain technology to the decentralization of energy?} 

\caption{Box 2: Flowchart steps 1-6: 1. Battery hub and converter; 2. connect app to battery hub; 3. Generate REC and ID; 4. Connect P2P trading; 5. Select energy type, quantity and how to sell, and 6. Receive payments in crypto or currency}



\section{Figures} 


\begin{figure}[htp]
\begin{center}
%\vspace{-1 in}
%\hspace{-0.75 in}\includegraphics[width=14cm,height=14cm]{PEB.eps}
\vspace{0.25 in}
\caption{Grid-interactive ebike system}
\vspace{2 in}
\end{center}
\end{figure}

%\caption{
%A. simplest
%B. network bikes -- battery -- (grid interactive stations -- home) -- grid
%C. network bikes -- battery -- deep (blockchain)  -- (grid interactive stations -- home) -- grid

%Flowchart of a decentralized E2GV model:
%  A. ebike hubs: Engineering: Status?
%  B. Token: Blockchain Ethereum or lamdem.io.
%  C. Dual energy power grid: algorithms and engineering.
%  D. Regulation and national boundaries: Velonetwork}
  
\caption{Fig. 2. Efficiency per capita energy production: kms-hub efficiency power-electricity plot + plugin cost : %loss V2G}


\section{FAQ} 

%\caption{FAQ}
How much energy can an ebike produce per time?\\
How much energy can an ebike network produce per time?\\ 
Which are the main scalability constraints?\\
How much energy can be sent back to the grid accounting for plugin losses, efficiency and regulations?\\


\section{Acknowledgements}


\section{References} 

\bibliographystyle{chicago}
\bibliography{references}

\end{document}



